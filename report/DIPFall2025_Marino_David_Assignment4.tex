\documentclass{article}
\usepackage{graphicx}
\usepackage{subcaption}
\usepackage{float}
\usepackage[margin=1in]{geometry}
\title{CS 4650/7650 ECE 4655/7655 Digital Image Processing 2025 Homework 4: Detection and Segmentation of the structures of interest}
\author{David Allen Stephan Marino}
\begin{document}
\maketitle
\newpage
\section{Experiments and Results}
\subsection{Blob Detection}
\subsubsection{Preprocessing}
\begin{figure}[htbp]
	\centering
	\begin{subfigure}{0.35\textwidth}
		\includegraphics[width=\linewidth]{../images/CancerCells-1.jpg}
		\caption{Cancer Cells Original}
	\end{subfigure}
	\begin{subfigure}{0.35\textwidth}
		\includegraphics[width=\linewidth]{../images/kidneytumorglomus.jpg}
		\caption{Kidney Tumor Glomus Original}
	\end{subfigure}
	\caption{Original Images}
\end{figure}
\begin{figure}[htbp]
	\centering
	\begin{subfigure}{0.35\textwidth}
		\includegraphics[width=\linewidth]{../images/cancer_gaussian.png}
		\caption{Cancer Cells Gaussian 11x11}
	\end{subfigure}
	\begin{subfigure}{0.35\textwidth}
		\includegraphics[width=\linewidth]{../images/kidney_gaussian.png}
		\caption{Kidney Tumor Gaussian 5x5}
	\end{subfigure}
	\caption{Gaussian Smoothed Images}
\end{figure}
\begin{figure}[htbp]
	\centering
	\begin{subfigure}{0.35\textwidth}
		\includegraphics[width=\linewidth]{../images/cancer_median.png}
		\caption{Cancer Cells Median 11}
	\end{subfigure}
	\begin{subfigure}{0.35\textwidth}
		\includegraphics[width=\linewidth]{../images/kidney_median.png}
		\caption{Kidney Tumor Glomus Median 5}
	\end{subfigure}
	\caption{Median Smoothed Images}
\end{figure}
\newpage
\subsubsection{Preprocessing Discussion}
For the cancer cell image, Gaussian filtering was applied to reduce the noise in the nuclei (green regions).
The Median filter produced far too much element merging.
The Gaussian filter reduced the noise within the nuclei and still upheld the integrity of the element's shape without melting into its neighbors.
\begin{figure}[htbp]
	\centering
	\begin{subfigure}{0.3\textwidth}
		\includegraphics[width=\linewidth]{../images/cancer_zoom.png}
		\caption{Zoom}
	\end{subfigure}
	\begin{subfigure}{0.3\textwidth}
		\includegraphics[width=\linewidth]{../images/cancer_gaussian_zoom.png}
		\caption{Zoom Gaussian 11x11}
	\end{subfigure}
	\begin{subfigure}{0.3\textwidth}
		\includegraphics[width=\linewidth]{../images/cancer_median_zoom.png}
		\caption{Zoom Median 11}
	\end{subfigure}
	\caption{Cancer Zoom Images}
\end{figure}
For the kidney cells, Median filtering was used to smooth the nuclei (purple regions).
The Median filter effectively removed noise within the nuclei while keeping the integrity of the elements' borders and shape.
The Gaussian filter, while also effectively upholding the integrity of the elements, was not as effective in reducing the noise present within the nuclei.
\begin{figure}[htbp]
	\centering
	\begin{subfigure}{0.3\textwidth}
		\includegraphics[width=\linewidth]{../images/kidney_zoom.png}
		\caption{Zoom}
	\end{subfigure}
	\begin{subfigure}{0.3\textwidth}
		\includegraphics[width=\linewidth]{../images/kidney_gaussian_zoom.png}
		\caption{Zoom Gaussian 5x5}
	\end{subfigure}
	\begin{subfigure}{0.3\textwidth}
		\includegraphics[width=\linewidth]{../images/kidney_median_zoom.png}
		\caption{Zoom Median 5}
	\end{subfigure}
	\caption{Kidney Zoom Images}
\end{figure}
\newpage

\subsection{Blob Detection}
\subsubsection{Cancer Blob Detection}
\begin{figure}[h]
	\centering
	\includegraphics[width=\linewidth]{../images/cancer_ggray.png}
	\caption{Grayscale Green Channel}
\end{figure}
Determined that the cancer cells' green channel best isolated the nuclei.
This removed much of the noise present in the other regions of the cells.
\newpage
\begin{figure}[h]
	\centering
	\includegraphics[width=\linewidth]{../images/cancer_LoG.png}
	\caption{Laplacian of Gaussian}
\end{figure}
At sigma 29.0, the nuclei of the cancer cells were best detected and isolated.
We apply LoG filter with sigma 5.0 because the elements of interest are relatively small.
However, melting is still seen in very dense regions.
Adjusting the shape of the nuclei using morphology techniques could prove helpful in further isolating these elements.
\newpage
\begin{figure}[h]
	\centering
	\includegraphics[width=\linewidth]{../images/cancer_binary.png}
	\caption{Binary Detection Mask}
\end{figure}
The binary detection mask does a good job of isolating the nuclei.
It reduces the area of the elements themselves but more accurately isolates the elements.
As the threshold for LoG increases, less merging is found but more false negatives.
As the threshold for LoG decreases, more merging is found but fewer false negatives.
The most accurate representation is found at a threshold value of 0.00075.
\newpage
\subsubsection{Kidney Blob Detection}
\begin{figure}[h]
	\centering
	\includegraphics[width=\linewidth]{../images/kidney_inverse.png}
	\caption{Grayscale Blue Channel Inversed}
\end{figure}
The kidney cells' imagery was mostly in the red spectrum without much contrast between channels.
Since the elements of interest are the purple nuclei, the blue channel presented itself as the best option.
This isolated the nuclei; however, since the nuclei are much lower in luminance, this meant that the LoG filter would pick up the background rather than the foreground.
So to fix this we invert the kidney imagery before passing in to the LoG filter.
\newpage
\begin{figure}[h]
	\centering
	\includegraphics[width=\linewidth]{../images/kidney_LoG.png}
	\caption{Laplacian of Gaussian}
\end{figure}
At sigma 31.0, the nuclei of the kidney cells were best detected and isolated.
The kidney cells are larger than the ones seen in the cancer cells so a larger sigma value is used.
The LoG filter accurately captures the elements with little to none anomalies.
As sigma value increases false merging is present.
As sigma value decreases false positives are found throughout.
\newpage
\begin{figure}[h]
	\centering
	\includegraphics[width=\linewidth]{../images/kidney_binary.png}
	\caption{Binary Detection Mask}
\end{figure}
Applying a binary mask to the LoG filter, we isolate the nuclei (purple regions) of the kidney cells.
The binary mask does present some false positives and false merges.
As mentioned before, the LoG favors the high luminance regions of the background rather than the foreground.
The most accurate representation is found at a threshold value of 0.0004.
\newpage
\subsection{K-Means Clustering}
\subsubsection{Cancer K-Means Clustering}
\begin{figure}[h]
	\centering
	\includegraphics[width=\linewidth]{../images/cancer_ggray.png}
	\caption{Grayscale Green Channel Gaussian 11x11}
\end{figure}
Determined that the cancer cells' green channel best isolated the nuclei.
This removed much of the noise present in the other regions of the cells.
\newpage
\begin{figure}[h]
	\centering
	\includegraphics[width=\linewidth]{../images/cancer_kmeans.png}
	\caption{K-Means 4 Clusters 4 Iterations 11}
\end{figure}
The cancer cells' imagery was segmented into 4 different clusters and iterated 11 times before convergence.
Below 4 clusters, there was not enough intermediary information, causing false merging and false negatives.
Above 4 clusters, there was too much information and false positives presenting themselves.
At 4 clusters, it presented the most accurate form with the least amount of false merging and false positives.
\newpage
\begin{figure}[h]
	\centering
	\includegraphics[width=\linewidth]{../images/cancer_kmeans_binary.png}
	\caption{Binary Detection Mask Threshold Lower Bound 200}
\end{figure}
To isolate the nuclei in the cancer cells, a high threshold value was used.
Since we segmented the cancer imagery into 4 clusters, thresholding was effective in creating a binary detection mask.
The cluster count did not affect the threshold lower boundary.
As the cluster count increased, the threshold value needed to be reduced to properly capture the elements.
When cluster count fell below 4, however, false positives and false merges were present, as there was not enough intermediary information for thresholding to be effective.
The results are satisfactory as there are few anomalies and accurately capture the nuclei elements of interest.
\newpage
\subsubsection{Kidney K-Means Clutering}
\begin{figure}[h]
	\centering
	\includegraphics[width=\linewidth]{../images/kidney_rgray.png}
	\caption{Grayscale Blue Channel Median 5}
\end{figure}
The kidney cells' imagery was mostly in the red spectrum without much contrast between channels.
Since the elements of interest are the purple nuclei, the blue channel presented itself as the best option.
\newpage
\begin{figure}[h]
	\centering
	\includegraphics[width=\linewidth]{../images/kidney_kmeans.png}
	\caption{K-Means Clusters 4 Iterations 15}
\end{figure}
K-Means was applied with 4 clusters and had 15 iterations before convergence.
Below 4 clusters, there was not enough intermediary information, causing false merging and false negatives.
Above 4 clusters, there was too much information, and false positives presented themselves.
At 4 clusters, it presented the most accurate form with the least amount of false merging and false positives.
\newpage
\begin{figure}[h]
	\centering
	\includegraphics[width=\linewidth]{../images/kidney_kmeans_binary.png}
	\caption{Binary Detection Mask Thresholding Lower Bound 150}
\end{figure}
Since the nuclear elements of interest were low in luminance, the image was inverted before thresholding.
In comparison to the cancer cells, the kidney cells required a lower thresholding value to capture the nuclear elements.
This is due to the color uniformity of the imagery, and a greater range needs to be captured to isolate the elements.
The results are not very satisfactory false positives are present but there are few anomalies and accurately captures the elements of interest.
\newpage
\begin{itemize}
	\item Dr. Filiz Bunyak\\
	      \textit{Blob Detection}
	\item Dr. Filiz Bunyak\\
	      \textit{Segmentation}
	\item OpenCV documentation\\
	      \textit{https://docs.opencv.org/4.x/}.
	\item PyPlot documentation\\
	      \textit{https://matplotlib.org/stable/tutorials/pyplot.html}
	\item Python documentation\\
	      \textit{https://docs.python.org/3/}
	\item JupyterLab documentation\\
	      \textit{https://jupyterlab.readthedocs.io/en/latest/}
	\item NumPy documentation\\
	      \textit{https://numpy.org/doc/}
\end{itemize}
\end{document}
